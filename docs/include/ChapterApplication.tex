\chapter{YART: A Ray Tracing Application} \label{ch:Application}

Proposed by this dissertation, \textit{Yet Another Ray Tracer} (YART) is a cross-platform, interactive 3D rendering application with an integrated ray tracing engine, made entirely on the CPU. 
It was build using the C++ programming language with a goal of exploring the inner workings of a ray tracing system, and testing the capabilities of modern CPUs.
As opposed to many state-of-the-art renderers which focus greatly on high performance and efficiency, YART is instead designed as a highly customizable and easily scalable environment. 
While not prioritizing the engines performance, treading is used in various parts of the system to utilize the power of CPU parallelization.

\section{Application Architecture}

threading, ray tracing done on CPU with pixel output (can be reused and used as an offline renderer) streamed to the GPU for presenting to the OS window, GPU for GUI rendering 

\dots

\section{Functionality Overview}

\dots

\section{External Dependencies}

\dots

\subsection{Vulkan}

\dots

\subsection{Dear ImGui}

\dots

\subsection{GLM}

\dots

\subsection{stb}

\dots

\section{Project Structure}

\dots

\subsection{Modularity}

\dots
