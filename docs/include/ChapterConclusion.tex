%--------------------------------------------------------------------------------------------
%   YART thesis "Conclusion" chapter definition.
%--------------------------------------------------------------------------------------------

\chapter{Conclusion} \label{ch:Conclusion}

This dissertation introduced YART, an interactive ray tracing application, capable of rendering complex, user-defined scenes on the CPU.
The most fundamental elements of a ray tracing engine have been explored and explained through an implementation process of a basic rendering procedure, described in a bottom-up approach.
Rendering of various geometric primitives has been accomplished by implementation of the M{\"o}ller-Trumbore algorithm, which enables highly efficient ray-triangle intersection testing.
Representation of geometric objects using triangle meshes allows the application to compactly store various, highly detailed objects of arbitrary shapes, such as cubes, spheres, or cones. 
Additionally, basic principles of simulating physically-based light propagation have been presented, employing the Blinn-Phong reflection model. 
Use of this shading model aided us in rendering diffuse materials with specular highlights, resulting in realistic, matte and reflective objects.
All of the described components were ultimately combined into a single ray tracing loop, which effectively defines YART's offline rendering engine. 

Despite evaluating all ray tracing computations on the CPU, YART remains a highly flexible, interactive application, which can render complex scenes in real-time.
Additionally, the modular and scalable nature of YART's architecture enables for further development of the system.
A number of advanced shading models, new types of light sources, materials, as well as visual effects such as depth of field, or anti-aliasing can be added to the engine, in order to produce more visually stunning and realistic images.
Then, in a case of potential deterioration of performance, a high amount of optimization techniques can be applied to the engine, such as bounding volume hierarchies (BVHs) \supercite{Wald2007}, which greatly accelerate the process of testing for intersections.  
