%--------------------------------------------------------------------------------------------
%   YART thesis abstract page definition.
%--------------------------------------------------------------------------------------------

\chapter*{abstract}
\addcontentsline{toc}{chapter}{ABSTRACT}

The purpose of this dissertation is to explore the world of ray tracing, focusing on fundamental principles of physically-based rendering and introducing a novel system called YART.
Designed to test the capabilities of modern CPUs, YART is an interactive 3D rendering application with an integrated ray tracing engine. 
The intricate creation process of such a system is described by delving into implementation detail of various key components of the YART engine, presented in a bottom-up approach.
Starting with the definition of a ray and ending in the implementation of a rendering loop, numerous algorithms are described along the way, such as ray-triangle intersection testing using M{\"o}ller-Trumbore, or the Blinn-Phong shading model. 

\vspace*{1em}

\noindent\textbf{Keywords:} Ray tracing, CPU, M{\"o}ller-Trumbore, Blinn-Phong

\clearpage
