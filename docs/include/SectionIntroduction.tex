\section{Introduction}

\subsection{Overview}

Vision is the single most advanced of human senses, which plays a vital role in determining how we perceive and experience the world.
It is not a surprise, that an ever-increasing amount of research and advancement has been made in pursuit of creating more photorealistic imagery in the field of computer graphics. 
In recent years, the dynamic landscape of computer graphics has witnessed a revolutionary shift with the rise and advancement of Ray Tracing technology. Unlike conventional rasterization techniques, Ray Tracing follows physical principles of light propagation, resulting in unparalleled visual realism. 

The purpose of this study is to explore and elucidate the intricate inner workings behind a Ray Tracing engine, delving into the underlying principles and various considerations involved in the creation of such a system.
(why CPU) \dots

\subsection{Ray Tracing}

\dots

\subsection{Existing implementations}

\dots

\subsection{Outline}

This introductory chapter highlights the key purpose of this dissertation, provides background, and gives a brief outline of existing solutions. 
Chapter 2 aims to \dots