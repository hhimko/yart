\chapter{Introduction}

Vision is the single most advanced of human senses, which plays a vital role in determining how we perceive and experience the world.
It is not a surprise, that an ever-increasing amount of research and advancement has been made in pursuit of creating more photorealistic imagery in the field of computer graphics. 
In recent years, the dynamic landscape of computer graphics has witnessed a revolutionary shift with the rise and advancement of Ray Tracing technology. 
Unlike conventional rasterization techniques, Ray Tracing follows physical principles of light propagation, resulting in unparalleled level of visual realism. 

The purpose of this paper is to explore and elucidate the intricate inner workings behind a Ray Tracing engine, delving into the underlying principles and various considerations involved in the creation of such a system.
As opposed to a more conventional approach of implementing Ray Tracing within a specialized GPU pipeline, the proposed solution has instead been developed entirely on the CPU. 
Significant differences between these two types of implementations are noteworthy, especially in how they handle parallel processing, which can impact the performance and efficiency of a Ray Tracing application. 
While the highly parallel nature of a GPU is often preferred for commercial, real-time rendering systems, a CPU side implementation allows for gaining a deep understanding of fundamental principles of computer graphics, without the added complexity of GPU programming.

\section{Ray Tracing}

\dots

\section{Existing implementations}

\dots

\section{Outline}

This introductory chapter highlights the key purpose of this dissertation, provides background, and gives a brief outline of existing solutions. 
Chapter 2 aims to \dots