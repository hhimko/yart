%--------------------------------------------------------------------------------------------
%   YART thesis "Implementation" chapter definition.
%--------------------------------------------------------------------------------------------

\chapter{Implementation} \label{ch:Implementation}

Before delving into the implementation of a ray tracing engine, certain considerations should be made, which will define the entire development workflow.
One such example is the choice of a coordinate system, that will affect the order of mathematical calculations.
Coordinate systems provide a standardized way of specifying locations and orientations of objects within a multi-dimensional environment.
In the context of two-dimensional computer graphics, the system typically consists of a X-axis pointing from left to right, and a Y-axis pointing upwards.
When adding the third coordinate however, the developer has a choice as to whether the Z-axis should point out of, or into the virtual screen, away from the viewer.
These two system types are respectively referred to as the right-handed coordinate system (RHS), and the left-handed coordinate system (LHS) (see \cref{fig:Implementation/coordinate_system}).
The names of these systems derive from the \textit{right-hand rule}, which is a convention used for determining the orientation of axes in three-dimensional space \supercite{RightHandRule}. 

\vfill
\begin{figure}[!ht]
    \centering

    \begin{subfigure}{.4\textwidth}
        \centering

        \begin{tikzpicture}[scale=1.15, every node/.style={scale=1.15}]
            \draw[-{Latex[length=3mm]}, color=axis_red, very thick]   (0, 0) -- (2, -0.7);
            \draw[-{Latex[length=3mm]}, color=axis_green, very thick] (0, 0) -- (0, 2);
            \draw[-{Latex[length=3mm]}, color=axis_blue, very thick]  (0, 0) -- (-1.6, -0.9);
            \node[color=axis_red] (x) at (2.05, -0.4) {\textbf{x}};
            \node[color=axis_green] (y) at (-0.3, 1.95) {\textbf{y}};
            \node[color=axis_blue] (y) at (-1.7, -0.6) {\textbf{z}};
        \end{tikzpicture}
        \caption{}
    \end{subfigure}%
    \begin{subfigure}{.4\textwidth}
        \centering
        
        \begin{tikzpicture}[scale=1.25, every node/.style={scale=1.25}]
            \draw[-{Latex[length=3mm]}, color=axis_red, very thick]   (0, 0) -- (2, -0.7);
            \draw[-{Latex[length=3mm]}, color=axis_green, very thick] (0, 0) -- (0, 2);
            \draw[-{Latex[length=3mm]}, color=axis_blue, very thick]  (0, 0) -- (1.6, 0.9);
            \node[color=axis_red] (x) at (2.05, -0.4) {\textbf{x}};
            \node[color=axis_green] (y) at (-0.3, 1.95) {\textbf{y}};
            \node[color=axis_blue] (y) at (1.5, 1.12) {\textbf{z}};
        \end{tikzpicture}
        \caption{}
    \end{subfigure}

    \caption[RHS and LHS coordinate systems]{RHS (a) and LHS (b) coordinate systems.}
    \label{fig:Implementation/coordinate_system}
\end{figure}
\vfill

Both the RHS and LHS are commonly utilized in commercial rendering software.
While developing YART, the left-handed coordinate system has been used, solely based on personal preference. 
This information is important to keep in mind, as the choice of a system determines the definition and implementation of projections, and various matrix operations, used throughout a ray tracing engine. 

\section{Ray Definition}

The fundamental piece of any ray tracing engine is undeniably the concept of \textit{rays}.
In mathematics, a ray can be defined as a straight line extending infinitely in one direction from a specific starting point. 
It is therefore characterized by its origin point, and a unit vector denoting the directions it's facing.
Rays can be conceptualized across any number of dimensions, however for the specific application of ray tracing, we will only focus on three-dimensional rays.

Let's consider a ray with an origin point $ \bm{P} $, extending infinitely in the direction of a vector $ \bm{\hat{v}} $.
To sample a specific point along this ray, we define a function $ \bm{r} $:
\begin{equation}
    \bm{r}(t) = \bm{P} + t\bm{\hat{v}}
\end{equation}
It returns all possible points on the ray, which are $ t $ units apart from its origin.
Knowing the distance from the ray's origin to the closest object hit along its path, function $ \bm{r} $ will enable us to find the exact point of collision on the object's surface.
For this purpose, it's important to make the direction vector $ \bm{\hat{v}} $ an unit vector, as a vector of length different than $ 1 $ would scale the distance proportionately, ultimately resulting in inaccurate calculations. 

In the code, we can define a ray as a lightweight structure containing two three-dimensional vector members, representing the ray's origin and direction (see \cref{lst:Implementation/RayDefinition/ray}).
This structure can later be extended with related operations, such as testing for ray-object intersections or reflecting of a given surface.

\vfill
\lstinputlisting[
    xleftmargin=1.5em, 
    caption={Ray structure definition.},
    label={lst:Implementation/RayDefinition/ray}
]{include/listings/ListingRay.cpp}
\vfill

\section{Ray Generation and The Camera}

To render images using ray tracing methodology, generally means to trace a single ray for every pixel in the output image.
The color seen it the direction of those rays, is what defines the final color of their respective pixels. 
This initial phase of calculating ray directions for every image pixel, is often referred to as the \textit{ray generation} step of a ray tracing engine.
Ray generation in YART can be divided into four stages:
\begin{enumerate}
    \item defining and querying camera properties,
    \item computing a screen-space-to-camera-space transformation matrix,
    \item calculating individual ray directions using the matrix,
    \item caching the calculated direction vectors.
\end{enumerate}

\subsection{Camera Definition}

Before we can generate a ray, we first need to explain the basic concept behind a \textit{camera}.
Sometimes referred to as the \textit{eye}, a camera is the engine's virtual viewpoint, giving the ability to "see" into a digital environment.
At its core, it's used to generate rays for each pixel in the rendered image.
It is primarily represented by its position in world-space, a viewing direction, and the camera's field of view (FOV).
A camera's field of view represents the angular extent of the scene captured in the rendered image. 
FOV might be expressed in one of three different ways, as it can be measured horizontally, vertically, or diagonally.
The difference between them significant, impacting the calculations made while defining a projection matrix.
In YART, horizontal measurement is commonly used to determine the field of view.

In a class defining a simplified camera model, we can introduce the functionality of calculating ray directions with a dedicated \verb|Camera::GetRayDirections| method. 
This method, can be declared to accept (as a parameter) an array of vectors, which will be populated with subsequent ray directions for each and every pixel in the output image.
Consequently, additional \verb|width| and \verb|height| parameters should be defined, which will specify the dimensions of the image, as well as the array's size.
\cref{lst:Implementation/RayGeneration/camera} demonstrates an example \verb|Camera| class declaration, constructed basing on the described methodology.

\vfill
\lstinputlisting[
    xleftmargin=2em, 
    caption={\texttt{Camera} class declaration.},
    label={lst:Implementation/RayGeneration/camera}
]{include/listings/ListingCamera1.cpp}
\vfill

\subsection{Transformation Matrices}

\subsection{Calculating Ray Directions}

\subsection{Ray Direction Caching}

YART employs a caching mechanism

Given the output image's pixel coordinates, the 

\section{Scene Representation}

mesh definition, normals calculation, uv coordinates, sphere mesh generation(?), light objects

\dots

\section{Intersection Testing, Materials, and Shading}

Möller-Trumbore algorithm for tri-ray intersection, solid color materials, shading using blinn-phong reflection model

\dots

\section{Sky Color Sampling}

solid color skies, linear gradients, and cubemap skyboxes

\dots

\section{Shadows}

hard shadows implementation

\dots

\section{Bounding Volume Hierarchies}

optimization strategies using BVH trees, AABB tree implementation

\dots
