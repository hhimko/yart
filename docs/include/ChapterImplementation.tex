%--------------------------------------------------------------------------------------------
%   YART thesis "Implementation" chapter definition.
%--------------------------------------------------------------------------------------------

\chapter{Implementation} \label{ch:Implementation}

Before delving into the implementation of a ray tracing engine, certain considerations should be made, which will define the entire system's workflow.
One such example is the choice of a world coordinate system, that will affect the order of mathematical calculations.

\vfill
\begin{figure}[!ht]
    \centering

    \begin{subfigure}{.4\textwidth}
        \centering

        \begin{tikzpicture}[auto]
            \draw[-{Latex[length=3mm]}, color=axis_red, very thick]   (0, 0) -- (2, -0.7);
            \draw[-{Latex[length=3mm]}, color=axis_green, very thick] (0, 0) -- (0, 2);
            \draw[-{Latex[length=3mm]}, color=axis_blue, very thick]  (0, 0) -- (-1.6, -0.9);
            \node[color=axis_red] (x) at (2.05, -0.4) {\textbf{x}};
            \node[color=axis_green] (y) at (-0.3, 1.95) {\textbf{y}};
            \node[color=axis_blue] (y) at (-1.7, -0.6) {\textbf{z}};
        \end{tikzpicture}
        \caption{}
    \end{subfigure}%
    \begin{subfigure}{.4\textwidth}
        \centering
        
        \begin{tikzpicture}[auto]
            \draw[-{Latex[length=3mm]}, color=axis_red, very thick]   (0, 0) -- (2, -0.7);
            \draw[-{Latex[length=3mm]}, color=axis_green, very thick] (0, 0) -- (0, 2);
            \draw[-{Latex[length=3mm]}, color=axis_blue, very thick]  (0, 0) -- (1.6, 0.9);
            \node[color=axis_red] (x) at (2.05, -0.4) {\textbf{x}};
            \node[color=axis_green] (y) at (-0.3, 1.95) {\textbf{y}};
            \node[color=axis_blue] (y) at (1.5, 1.12) {\textbf{z}};
        \end{tikzpicture}
        \caption{}
    \end{subfigure}

    \caption[RHS and LHS coordinate systems]{RHS (a) and LHS (b) coordinate systems.}
    \label{fig:Implementation/coordinate_system}
\end{figure}
\vfill

\section{Ray Definition}

mathematical definition of rays

\dots

\section{The Camera and Ray Generation}

camera sometimes referred to as "eye", view and projection matrixes

\dots

\section{Scene Representation}

mesh definition, normals calculation, uv coordinates, sphere mesh generation(?), light objects

\dots

\section{Intersection Testing, Materials, and Shading}

Möller-Trumbore algorithm for tri-ray intersection, solid color materials, shading using blinn-phong reflection model

\dots

\section{Sky Color Sampling}

solid color skies, linear gradients, and cubemap skyboxes

\dots

\section{Shadows}

hard shadows implementation

\dots

\section{Bounding Volume Hierarchies}

optimization strategies using BVH trees, AABB tree implementation

\dots
