%--------------------------------------------------------------------------------------------
%   YART thesis "Introduction" chapter definition.
%--------------------------------------------------------------------------------------------

\chapter{Introduction} \label{ch:Introduction}

Vision is the most advanced of human senses, which plays a vital role in determining how we perceive and experience the world.
It is not a surprise, that an ever-increasing amount of research and advancement has been made in pursuit of creating more photorealistic imagery in the field of computer graphics. 
In recent years, the dynamic landscape of computer graphics has witnessed a revolutionary shift with the rise and advancement of ray tracing technology. 
Unlike conventional rasterization methods, ray tracing follows physical principles of light propagation, resulting in unparalleled levels of visual realism. 

This dissertation aims to explore and elucidate the intricate inner workings behind a ray tracing engine, delving into the underlying principles and various considerations involved in the creation of such a system.
In contrast to a more conventional approach of implementing ray tracing within a specialized graphics processing unit (GPU) pipeline, the proposed solution has instead been developed entirely on the central processing unit (CPU). 
Significant differences between these two types of implementations are noteworthy, especially in how they handle parallel processing, which can impact the performance and efficiency of a ray tracing application. 
While the highly parallel nature of a GPU is often preferred for commercial, real-time rendering systems, a CPU side implementation allows for gaining a deep understanding of fundamental principles of computer graphics, without the added complexity of GPU programming.

\section{Ray Tracing}

The concept of ray tracing has gained increased interest in most recent years, having seen a convergence of various advancements in hardware, algorithms, and an ever-growing industry demand for realism.
Its origin however, dates back to as far as 16th century, when it was first described by a german artist and theorist Albrecht Dürer \supercite{Hofmann1990}.
At its core, ray tracing is a graphics rendering technique, used to simulate the way light interacts with objects within a virtual environment. 
The process involves scattering rays of light from a virtual camera, tracing their paths as they travel and interact with a scene. 
Individual light rays are traced "backward", originating from the camera's viewpoint, as opposed to how they would be cast from a light source in a more realistic setting.
Doing so however, is many orders of magnitude more optimal, since only a small percent of rays emitted from a given light source actually make it directly into the camera's lens and contribute to the final image \supercite{Glassner1989}.
Each ray is then projected through a pixel on the output image plane, determining it's color by testing for intersections and sampling materials of hit objects.
This basic idea behind ray tracing is illustrated in \cref{fig:Introduction/RayTracing/rt1}. 

\vfill
\begin{figure}[!ht]
    \centering

    % Path that follows the edges of the current page for clipping
    \tikzstyle{reverseclip}=[insert path={(current page.north east) --
        (current page.south east) --
        (current page.south west) --
        (current page.north west) --
        (current page.north east)}
    ]

    \pgfdeclareradialshading{tikzfadeout}{\pgfpointorigin}{
        color(1.5mm)=(pgftransparent!0); 
        % color(4mm)=(pgftransparent!20); 
        color(9mm)=(pgftransparent!100)
    }
    \pgfdeclarefading{fadeout}{\pgfuseshading{tikzfadeout}}

    \begin{tikzpicture}[scale=1.0, every node/.style={scale=1.0}]
        % Plane
        \draw[{Round Cap[length=40pt]}-{Round Cap[length=20pt]}, color=darkgray, very thick] 
            (12.7mm, -55mm) -- (-11.5mm, -40.5mm) -- (53mm, -20.5mm) -- (133mm, -44mm) -- (118mm, -55mm);

        % Camera Lens
        \draw[darkgray, thick] (0,0) ellipse (3mm and 3.6mm);
        \draw[darkgray] (-4.9mm,4.45mm) arc (105:260:3mm and 3.67mm);

        % Camera Body
        \draw[-, color=darkgray, thick] (-0.7mm, -3.5mm) -- (-4.6mm, -2.8mm) -- (-8.7mm, -3.4mm) -- 
            (-8.9mm, 4mm) -- (0.6mm, 5.3mm) -- (-3mm, 6.1mm) -- (-12.7mm, 4.8mm)  -- (-12.4mm, -2.3mm) --
            (-8.7mm, -3.4mm) -- (-8.9mm, 4mm) -- (-12.7mm, 4.8mm);
        \draw[-, color=darkgray, thin] (0.6mm, 3.62mm) -- (-4.7mm, 4.55mm);
        \draw[-, color=darkgray, thick] (1.2mm, 3.4mm) -- (1.2mm, 5.3mm);

        % Sphere
        \node[draw=axis_blue!20!darkgray, very thick, fill=axis_blue!65, circle, inner sep=0pt, minimum size=36mm] (c) at (79.6mm, -20mm) {};

        % Rays background
        \draw[{Round Cap[]}-, color=axis_red, very thick] (0, 0) -- (24.5mm, -0.95mm);
        \draw[{Round Cap[]}-, color=axis_red, very thick] (0, 0) -- (29mm, -18.34mm);

        % Image Plane fill
        \draw[-, color=white!0, fill=white, fill opacity=0.65] (6.8mm, -27.5mm) -- (6.2mm, 6.2mm) --
            (44mm, 11mm) -- (43.3mm, -17.5mm) -- (6.8mm, -27.5mm) -- (6.2mm, 6.2mm);

        \node[draw=axis_blue!20!darkgray, thin, fill=axis_blue!65, circle, inner sep=0pt, minimum size=16mm, rotate=90, xslant=-0.1, fill opacity=0.65, draw opacity=0.65] (c) at (26.3mm, -7mm) {};

        % Image Plane inner lines 
        \draw[-, color=gray, thin] (12mm, -26mm) -- (11.6mm, 6.8mm);
        \draw[-, color=gray, thin] (17.3mm, -24.7mm) -- (16.9mm, 7.4mm);
        \draw[-, color=gray, thin] (22mm, -23.5mm) -- (21.9mm, 8mm);
        \draw[-, color=gray, thin] (26.8mm, -22.2mm) -- (26.7mm, 8.6mm);
        \draw[-, color=gray, thin] (31.2mm, -21mm) -- (31.2mm, 9.2mm);
        \draw[-, color=gray, thin] (35.2mm, -19.5mm) -- (35.5mm, 9.8mm);
        \draw[-, color=gray, thin] (39.5mm, -18.5mm) -- (39.8mm, 10.4mm);

        \draw[-, color=gray, thin] (6.8mm, -20.6mm) -- (43.5mm, -12mm);
        \draw[-, color=gray, thin] (6.7mm, -14mm) -- (43.5mm, -6.3mm);
        \draw[-, color=gray, thin] (6.5mm, -7.6mm) -- (43.6mm, -0.4mm);
        \draw[-, color=gray, thin] (6.2mm, -0.2mm) -- (44mm, 5.2mm);

        % Image Plane outer lines 
        \draw[-, color=darkgray, very thick] (6.8mm, -27.5mm) -- (6.2mm, 6.2mm) -- (44mm, 11mm) -- 
            (43.3mm, -17.5mm) -- (6.8mm, -27.5mm) -- (6.2mm, 6.2mm);

        % Rays foreground
        \draw[-{Latex[length=3mm]}, color=axis_red, very thick] (29mm, -18.34mm) -- (72.5mm, -45.8mm);
        \draw[-{Latex[length=3mm]}, color=axis_red, very thick] (24.5mm, -0.95mm) -- (73.5mm, -2.9mm);

        % Labels
        \node[color=darkgray] (x) at (-7mm, 10mm) {\textbf{Camera}};
        \node[color=axis_red,rotate=-2] (x) at (57mm, 1mm) {\textbf{Ray}};
        \node[color=darkgray, rotate=6, xslant=0.15] (x) at (25mm, 12mm) {\textbf{Image}};
        \node[color=axis_blue!80, align=left] (x) at (103mm, -8mm) {\textbf{Scene} \\ \textbf{\;\;Object}};

    \end{tikzpicture}

    \caption[Simplified ray tracing model]{Simplified ray tracing model.}
    \label{fig:Introduction/RayTracing/rt1}
\end{figure}
\vfill

When colliding with an object, the ray gets reflected of the surface, refracted, or absorbed depending on the hit object's material. 
Reflected rays are traced recursively to calculate the final color of a particular pixel, until being absorbed or reaching a certain limit of bounces.  

By following the physically-based principles of light, ray tracing enables straightforward simulation of highly realistic optical effects, such as refraction, depth of field and soft shadows.
This is in contrast to the traditional rendering method of rasterization, where producing similar results could be a difficult, or even an impossible task. 
The unmatched visual realism of ray tracing however does not come without cost. 
Ray tracing generally has a much greater computational cost, as compared to scanline rendering methods, which makes it a less performant algorithm of the two.
Until the late 2010s, creating an interactive ray tracing application for consumer hardware was generally considered as impossible.
Today, the ongoing development of commercial hardware and algorithmic advancements have made real-time ray tracing become feasible, making it a viable option for film production, video games, and commercial graphics applications.

\section{Related Work}

Numerous ray tracing engines and application programming interfaces (APIs) have been proposed for professional and commercial use since the dawn of physically-based computer graphics.
Each engine, having its distinct capabilities, purpose, and inherent limitations, has contributed greatly to the advancement of computer graphics. 
Following, is a list of noteworthy entries in the vast catalogue of existing ray tracing systems.

\subsection{NVIDIA OptiX}

First introduced in 2010, OptiX \supercite{Parker2010} is a programmable ray tracing framework developed for NVIDIA GPUs.
It has been designed with focus on both high flexibility and performance, targeting highly parallel hardware architectures.
The core idea behind the OptiX engine, is dividing the rendering pipeline into a small set of programmable operations or \textit{shaders}, akin to how rasterization pipelines are deployed within OpenGL and Direct3D applications.
These user-specified shaders can be defined to form a variety of ray tracing-based systems, including offline rendering, collision detection systems, and scientific simulations such as sound propagation.
For representing the scene, OptiX employs a lightweight data structure, called the \textit{context}, used for binding shaders to the object-specific data they require.
The context bundles together shaders, geometry data, and acceleration structures in the form of connected graph nodes, resulting in high flexibility and reusability.
While OptiX is a powerful tool for scientific research and professional ray tracing development, it may not be as user-friendly for artists and designers looking for out-of-the-box solutions.

\subsection{Intel Embree}

Embree \supercite{Wald2014}, is an open-source ray tracing framework developed at Intel, targeting x86 CPUs. 
Since its first release in 2011, it has become a cornerstone for a variety of leading 3D rendering applications. 
In contrast to NVIDIA's OptiX, which was designed to simplify development of new renderers, Embree aims to accelerate ray tracing in existing systems. 
This key design goal leads to highly optimized low-level kernels for accelerating ray generation and traversal that leverage modern CPU architectures.
All ray tracing operations supported by the Embree kernels can be accessed through an easy to use and compact API.
It exposes a number of functions for defining custom geometry and issuing intersection queries. 
The API aims to hide implementation detail such as data structure memory layout, while ensuring high performance intersection testing. 
Embree's lightweight and flexible API makes it in invaluable tool for developers seeking efficient ray tracing capabilities in their applications, without needing to worry about implementing lower-level optimizations and acceleration structures.

\subsection{Cycles}

Developed by the Blender Foundation, Cycles \supercite{Cycles} is an open-source ray tracing renderer, which can be integrated into commercial software. 
Since 2011, Cycles has been a part of the Blender 3D creation suite, and Blender's first physically-based rendering method.
It employs a \textit{path tracing} algorithm, which is a probability-based type of ray tracing.
Due to the unbiased nature and accuracy of the selected method, Cycles is able to naturally simulate highly realistic lighting scenarios, such as global illumination, soft shadows, and caustics.
Path tracing, is however prone to generating noise and requires a high number of samples to be made per pixel, until it can produce a relatively noise-free output.
In order to minimize the number of samples, while still ensuring accurate results, various denoising algorithms have been integrated into the engine, that aid in reducing noise artifacts in rendered images.
One of Cycles' standout features is its node-based material system, which provides users with a flexible and intuitive means of crafting advanced, physically accurate materials. 
Additionally, it supports GPU acceleration, which greatly improves the engine's performance, using the power of modern hardware.
The collaborative and open-source nature of Blender makes Cycles is a compelling choice for digital artists and designers seeking for a robust and accessible ray tracing solution.

\section{Outline}

This introductory chapter highlights the key purpose of this dissertation, provides background, and gives an outline of existing solutions. 
\cref{ch:Application} serves as a high-level overview of the proposed application, briefly describing its capabilities and architectural design. 
\cref{ch:Implementation} delves into implementation detail and various considerations involved with development of the ray tracing engine. 
The implementation has been divided into a set of smaller iterations, presented in an easy to follow, bottom-up approach.
Finally, \cref{ch:Conclusion} gives a brief conclusion, and describes possible future work related to the presented application.
